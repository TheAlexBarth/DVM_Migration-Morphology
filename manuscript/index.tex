% Options for packages loaded elsewhere
\PassOptionsToPackage{unicode}{hyperref}
\PassOptionsToPackage{hyphens}{url}
\PassOptionsToPackage{dvipsnames,svgnames,x11names}{xcolor}
%
\documentclass[
]{article}

\usepackage{amsmath,amssymb}
\usepackage{lmodern}
\usepackage{setspace}
\usepackage{iftex}
\ifPDFTeX
  \usepackage[T1]{fontenc}
  \usepackage[utf8]{inputenc}
  \usepackage{textcomp} % provide euro and other symbols
\else % if luatex or xetex
  \usepackage{unicode-math}
  \defaultfontfeatures{Scale=MatchLowercase}
  \defaultfontfeatures[\rmfamily]{Ligatures=TeX,Scale=1}
\fi
% Use upquote if available, for straight quotes in verbatim environments
\IfFileExists{upquote.sty}{\usepackage{upquote}}{}
\IfFileExists{microtype.sty}{% use microtype if available
  \usepackage[]{microtype}
  \UseMicrotypeSet[protrusion]{basicmath} % disable protrusion for tt fonts
}{}
\makeatletter
\@ifundefined{KOMAClassName}{% if non-KOMA class
  \IfFileExists{parskip.sty}{%
    \usepackage{parskip}
  }{% else
    \setlength{\parindent}{0pt}
    \setlength{\parskip}{6pt plus 2pt minus 1pt}}
}{% if KOMA class
  \KOMAoptions{parskip=half}}
\makeatother
\usepackage{xcolor}
\setlength{\emergencystretch}{3em} % prevent overfull lines
\setcounter{secnumdepth}{-\maxdimen} % remove section numbering
% Make \paragraph and \subparagraph free-standing
\ifx\paragraph\undefined\else
  \let\oldparagraph\paragraph
  \renewcommand{\paragraph}[1]{\oldparagraph{#1}\mbox{}}
\fi
\ifx\subparagraph\undefined\else
  \let\oldsubparagraph\subparagraph
  \renewcommand{\subparagraph}[1]{\oldsubparagraph{#1}\mbox{}}
\fi


\providecommand{\tightlist}{%
  \setlength{\itemsep}{0pt}\setlength{\parskip}{0pt}}\usepackage{longtable,booktabs,array}
\usepackage{calc} % for calculating minipage widths
% Correct order of tables after \paragraph or \subparagraph
\usepackage{etoolbox}
\makeatletter
\patchcmd\longtable{\par}{\if@noskipsec\mbox{}\fi\par}{}{}
\makeatother
% Allow footnotes in longtable head/foot
\IfFileExists{footnotehyper.sty}{\usepackage{footnotehyper}}{\usepackage{footnote}}
\makesavenoteenv{longtable}
\usepackage{graphicx}
\makeatletter
\def\maxwidth{\ifdim\Gin@nat@width>\linewidth\linewidth\else\Gin@nat@width\fi}
\def\maxheight{\ifdim\Gin@nat@height>\textheight\textheight\else\Gin@nat@height\fi}
\makeatother
% Scale images if necessary, so that they will not overflow the page
% margins by default, and it is still possible to overwrite the defaults
% using explicit options in \includegraphics[width, height, ...]{}
\setkeys{Gin}{width=\maxwidth,height=\maxheight,keepaspectratio}
% Set default figure placement to htbp
\makeatletter
\def\fps@figure{htbp}
\makeatother
\newlength{\cslhangindent}
\setlength{\cslhangindent}{1.5em}
\newlength{\csllabelwidth}
\setlength{\csllabelwidth}{3em}
\newlength{\cslentryspacingunit} % times entry-spacing
\setlength{\cslentryspacingunit}{\parskip}
\newenvironment{CSLReferences}[2] % #1 hanging-ident, #2 entry spacing
 {% don't indent paragraphs
  \setlength{\parindent}{0pt}
  % turn on hanging indent if param 1 is 1
  \ifodd #1
  \let\oldpar\par
  \def\par{\hangindent=\cslhangindent\oldpar}
  \fi
  % set entry spacing
  \setlength{\parskip}{#2\cslentryspacingunit}
 }%
 {}
\usepackage{calc}
\newcommand{\CSLBlock}[1]{#1\hfill\break}
\newcommand{\CSLLeftMargin}[1]{\parbox[t]{\csllabelwidth}{#1}}
\newcommand{\CSLRightInline}[1]{\parbox[t]{\linewidth - \csllabelwidth}{#1}\break}
\newcommand{\CSLIndent}[1]{\hspace{\cslhangindent}#1}

\usepackage{hyphenat} % don't hyphenat titles
\usepackage{graphicx}
\usepackage{lineno}\linenumbers


% Add any tex header commands here
\makeatletter
\makeatother
\makeatletter
\makeatother
\makeatletter
\@ifpackageloaded{caption}{}{\usepackage{caption}}
\AtBeginDocument{%
\ifdefined\contentsname
  \renewcommand*\contentsname{Table of contents}
\else
  \newcommand\contentsname{Table of contents}
\fi
\ifdefined\listfigurename
  \renewcommand*\listfigurename{List of Figures}
\else
  \newcommand\listfigurename{List of Figures}
\fi
\ifdefined\listtablename
  \renewcommand*\listtablename{List of Tables}
\else
  \newcommand\listtablename{List of Tables}
\fi
\ifdefined\figurename
  \renewcommand*\figurename{Figure}
\else
  \newcommand\figurename{Figure}
\fi
\ifdefined\tablename
  \renewcommand*\tablename{Table}
\else
  \newcommand\tablename{Table}
\fi
}
\@ifpackageloaded{float}{}{\usepackage{float}}
\floatstyle{ruled}
\@ifundefined{c@chapter}{\newfloat{codelisting}{h}{lop}}{\newfloat{codelisting}{h}{lop}[chapter]}
\floatname{codelisting}{Listing}
\newcommand*\listoflistings{\listof{codelisting}{List of Listings}}
\makeatother
\makeatletter
\@ifpackageloaded{caption}{}{\usepackage{caption}}
\@ifpackageloaded{subcaption}{}{\usepackage{subcaption}}
\makeatother
\makeatletter
\@ifpackageloaded{tcolorbox}{}{\usepackage[many]{tcolorbox}}
\makeatother
\makeatletter
\@ifundefined{shadecolor}{\definecolor{shadecolor}{rgb}{.97, .97, .97}}
\makeatother
\makeatletter
\@ifpackageloaded{sidenotes}{}{\usepackage{sidenotes}}
\@ifpackageloaded{marginnote}{}{\usepackage{marginnote}}
\makeatother
\makeatletter
\makeatother
\ifLuaTeX
  \usepackage{selnolig}  % disable illegal ligatures
\fi
\IfFileExists{bookmark.sty}{\usepackage{bookmark}}{\usepackage{hyperref}}
\IfFileExists{xurl.sty}{\usepackage{xurl}}{} % add URL line breaks if available
\urlstyle{same} % disable monospaced font for URLs
\hypersetup{
  pdftitle={Size and transparency influence diel vertical migration patterns in copepods.},
  pdfauthor={Alex Barth*; Rod Johnson; Joshua Stone},
  colorlinks=true,
  linkcolor={blue},
  filecolor={Maroon},
  citecolor={Blue},
  urlcolor={Blue},
  pdfcreator={LaTeX via pandoc}}

\title{Size and transparency influence diel vertical migration patterns
in copepods.}
\author{Alex Barth* \and Rod Johnson \and Joshua Stone}
\date{}

\begin{document}
    \begin{titlepage}
% This is a combination of Pandoc templating and LaTeX
% Pandoc templating https://pandoc.org/MANUAL.html#templates
% See the README for help

\raggedleft % left align the title page
\rule{1pt}{\textheight} % Vertical line
\hspace{0.05\textwidth} % Whitespace between the vertical line and title page text
% Adjust num before \textwidth to move the block left or right
\begin{minipage}[b][\textheight][s]{0.85\textwidth}

\raggedright
% Title and subtitle
{\large\bfseries\nohyphens{Size and transparency influence diel vertical
migration patterns in copepods.}}\\[2\baselineskip] 

  
  % Authors	
% This hairy bit of code is just to get "and" between the last 2
% authors. See below if you don't need that
 {\large{Alex Barth*}}{\textsuperscript{1}}%
%
, 
 {\large{Rod Johnson}}{\textsuperscript{2}}%
%
%
{ and \large{Joshua Stone}}%
{\textsuperscript{1}}%
%


% This is how to do it if you don't need the "and"

  
  %%%%%% Affiliations
\vspace{2\baselineskip} 

\hangindent=1em
\hangafter=1
%
{1}.~{University of South Carolina}%
%
, %
{Biological Sciences}%
%
%
, %
{700 Sumter St 401, Columbia, SC 29208}%
%
\par\hangindent=1em\hangafter=1%
%
{2}.~{Bermuda Institute of Ocean Sciences}%
%
%
, %
{Bermuda GE 01, UK}%
%

  
  %%%%%% Correspondence
\vspace{1\baselineskip} 

              
  %use \vfill instead to get the space to fill flexibly	
%\vspace{0.25\textheight} % Whitespace between the title block and the publisher

\vfill

%%%%%% Cover image

  
  % Whitespace between the title block and the tagline
\vspace{0.1\textheight} 

\end{minipage}  \end{titlepage}
  \ifdefined\Shaded\renewenvironment{Shaded}{\begin{tcolorbox}[interior hidden, boxrule=0pt, sharp corners, enhanced, frame hidden, borderline west={3pt}{0pt}{shadecolor}, breakable]}{\end{tcolorbox}}\fi

\setstretch{2}
\hypertarget{scientific-significance-statement}{%
\subsection{\texorpdfstring{\emph{Scientific Significance
Statement}}{Scientific Significance Statement}}\label{scientific-significance-statement}}

\hypertarget{study-novelty}{%
\subsubsection{Study Novelty}\label{study-novelty}}

Diel Vertical Migration is a widespread phenomenon across marine and
freshwater systems. The predator evasion hypothesis suggests that DVM
occurs as zooplankton attempt to escape visual predators. Yet, DVM
itself is a costly and risky behavior. Thus, DVM should only occur when
visual risk is high. Several studies have shown that copepod size
influences the magnitude of DVM. However, an individual's visual risk
may include traits beyond simply size. In this study, we utilize an
in-situ imaging tool to reveal how copepod morphological traits
influence DVM. Our findings show that both size and transparency
influence DVM. We support this finding through rigorous statistical
analyses and state-of-the-art technology. This finding provides support
for leading DVM hypotheses and highlights that DVM is a complex behavior
driven by multiple copepod traits. Furthermore, this study represents a
novel application of in-situ imaging technology to address major
hypotheses in biological oceanography.

\hypertarget{applicability-to-lo}{%
\subsubsection{Applicability to L\&O}\label{applicability-to-lo}}

This study addresses diel vertical migration, an active, major research
topic in biological oceanography. Many studies published in L\&O
contribute to advancing knowledge on DVM. In this paper, we provide
strong evidence for both size and transparency influencing DVM behavior.
Additionally, we accomplished this study using emerging technology and
statistical analyses. This work builds on research published in L\&O and
will be broadly applicable to plankton ecologists, biological
oceanographers.

\hypertarget{scientific-significance-statement-1}{%
\subsection{\texorpdfstring{\emph{Scientific Significance
Statement}}{Scientific Significance Statement}}\label{scientific-significance-statement-1}}

AB and JS developed the study hypotheses. JS coordinated deployment and
data management of the UVP. RJ facilitated data collection on cruises.
AB led the analysis and preparation of the manuscript and figures. JS
and RJ contributed to the manuscript draft. All authors approved the
final submission.

\hypertarget{abstract}{%
\section{Abstract}\label{abstract}}

Diel vertical migration (DVM) is a widespread phenomenon in aquatic
environments. The primary hypothesis explaining DVM is the
predation-avoidance hypothesis, which suggests that zooplankton migrate
to deeper waters to avoid detection during daylight. Copepods are the
predominant mesozooplankton undergoing these migrations, however they
display massive morphological variation. Visual risk also depends on a
copepod's morphology. In this study, we investigate hypotheses related
to morphology and DVM: (H1) as size increases visual risk, increases in
body size will increase DVM magnitude and (H2) if copepod transparency
can reduce visual risk, increases in transparency will reduce DVM
magnitude. In-situ copepod images were collected across several cruises
in the Sargasso Sea using an Underwater Vision Profiler 5. Copepod
morphology was characterized from these images and a dimension reduction
approach. While in-situ imaging offers challenges for quantifying
mesozooplankton behavior, we introduce a robust method for quantifying
DVM. The results show a clear relationship in which larger copepods have
a larger DVM signal. Darker copepods also have a larger DVM signal,
however only amongst the largest group of copepods and not smaller ones.
These findings highlight the complexity of copepod morphology and DVM
behavior.

\hypertarget{introduction}{%
\section{Introduction}\label{introduction}}

Diel vertical migration (DVM) is a wide spread phenomena with large
consequences in ocean ecosystems. DVM is the process of pelagic
organisms vertically moving in the water column on a daily basis, often
travelling dozens to hundreds of meters (Bianchi and Mislan 2016). This
large-scale event occurs across many taxa, from plankton to fish
(Brierley 2014). However, DVM is particularly notable in zooplankton
communities, whose migrations contribute substantially to biogeochemical
cycles (Steinberg and Landry 2017; Archibald et al. 2019; Siegel et al.
2023). Mesozooplankton communities, largely dominated by copepods
(Turner 2004), will feed in surface layers of the ocean at night then
migrate into deeper waters during daytime. Through this movement,
copepods actively transport carbon to depth. Additionally, Kelly et al.
(2019) described zooplankton DVM to be a major component of mesopelagic
food webs. Thus to understand pelagic food webs and nutrient cycles, it
is critically important to understand the drivers of DVM.

DVM has long been studied in marine systems (Bandara et al. 2021).
Predominantly, zooplankton DVM is the movement from deep waters at
daytime to shallower waters at night (Hays 2003; Bianchi and Mislan
2016). However, reverse migration is also well documented (Ohman 1990).
The adaptive benefits of DVM have been extensively reviewed (Lampert
1989; Hays 2003; Cohen and Forward Jr. 2009; Ringelberg 2009; Williamson
et al. 2011; Bandara et al. 2021). Some studies have hypothesized that
DVM provides a physiological advantage. It has been suggested that
moving to deeper waters may provide zooplankton a reduction in UV-damage
(Ewald 1912; Kessler et al. 2008), metabolic benefits (McLaren 1963;
Enright 1977), or demographic benefits (McLaren 1974). However, the
predator-avoidance hypothesis has received the most support to explain
ultimate causes of DVM (see review of current evidence by Bandara et al.
2021). First described by Zaret and Suffern (1976), this hypothesis
posits zooplankton evacuate the sunlit surface to evade visual predators
then ascend at night to feed. However the massive migration undertaken
by zooplankton is energetically expensive (Maas et al. 2018; Robison et
al. 2020). Therefore, the predator-avoidance hypothesis makes a clear
prediction that the trade-off of expended energy is worth the predator
avoidance benefit (Lampert 1989). This trade-off has been further
described in observations of the relationship between zooplankton
feeding and DVM patterns which led to the hunger-satiation hypothesis
(Atkinson et al. 1992; Pearre 2003). This hypothesis suggests that
vertical migrators will ascend to feed when hungry then retreat once
full. Once an individual has fully fed, remaining at the surface
provides no benefit while their visual risk may increase due to their
full guts which may increase visibility. Thus, the hunger-satiation
hypothesis provides a detailed case of the predator-avoidance hypothesis
and suggests cases where copepods may forego DVM. Regardless, both the
hunger-satiation hypothesis and the predator-avoidance hypothesis
suggest DVM is primarily a result of top-down control. In modelling
studies with copepods, the predominant oceanic zooplankton, top-down
control (Bandara et al. 2019) and trophic interactions (Pinti et al.
2019) have successfully been used to replicate DVM patterns.

Predator-driven migration suggests that DVM can be a function of an
individual copepod's detection risk by a visual predator. However, this
risk can depend on a copepod's morphological features (Aksnes and Utne
1997). Notably a copepod's size can increase visual detection. Several
studies have documented that copepod size influences DVM magnitude (Hays
et al. 1994; Ohman and Romagnan 2016; Aarflot et al. 2019). Presumably,
a copepod's transparency will also influence DVM. Hays et al. (1994)
reported that pigmentation explained variation in DVM frequency.
However, few other studies have investigated this at length. One barrier
to studying a relationship between copepod morphology and DVM is the
difficulty of accurately recording traits. Several approaches have been
utilized to study DVM. High spatiotemporal resolution of DVM can be
achieved through acoustic (Liu et al. 2022), and even satellite-based
measurements (Behrenfeld et al. 2019). However these approaches do not
yield information about individuals, much less traits. Net collected
specimens can allow for trait-related investigations of copepod DVM
patterns (Hays et al. 1994; Ohman and Romagnan 2016). However, it is
much more challenging to measure traits related to copepod transparency
from net-collected specimens. Copepods collected from deep net tows can
be severely damaged and their gut contents may not reflect natural
conditions due to cod-end feeding or regurgitation. Furthermore, typical
preservation methods of net-specimens can result in the loss of
pigmentation through bleaching in ethanol or formalin or increases in
opacity as the copepod dies. Yet traits related to copepod's
transparency are not well captured in net-collected specimens which may
evacuate gut contents or lose pigmentation following preservation in
formalin or ethanol. In Hays et al. (1994)'s investigation, the authors
relied on previously published copepod carotenoid values in their
analyses rather than attempt to measure pigment values from their
preserved specimens.

However, these sampling challenges may be effectively circumvented with
the emerging use of in-situ imaging tools. By directly observing
copepods, new insights into their behavior and traits can be resolved
(Ohman 2019). For example, Whitmore and Ohman (2021) used an in-situ
imaging device to describe a relationship between copepod abundance with
a particulate field rather than chlorophyll-a. Such findings are
facilitated by the fact imagery data records an individual's exact
position. Additionally, a copepod's true appearance, including difficult
to record metrics like transparency, can be measured. Thus, in-situ
imaging offers a new perspective to investigate DVM hypotheses. Some
studies observed a copepod DVM pattern with in-situ imagery data (Pan et
al. 2018; Whitmore and Ohman 2021). However, direct tests of DVM-related
hypotheses with such data have not yet been conducted.

In this study, we utilized in-situ imaging to evaluate how copepod
morphological traits influence DVM patterns. We specifically test the
hypotheses that, (H1) as size increases visual risk, increases in body
size will increase DVM magnitude and (H2) if copepod transparency can
reduce visual risk, increases in transparency will reduce DVM magnitude.
If these morphologically based hypotheses are true, then the larger and
darker copepods will have the largest DVM magnitude.

\hypertarget{methods}{%
\section{Methods}\label{methods}}

\hypertarget{ctd-profiles-and-uvp-imaging-of-copepods}{%
\subsection{CTD profiles and UVP imaging of
copepods}\label{ctd-profiles-and-uvp-imaging-of-copepods}}

Data were collected aboard the R/V Atlantic Explorer in collaboration
with the Bermuda Atlantic Time-series Study (BATS) (Steinberg et al.
2001). In-situ images of plankton were acquired using an Underwater
Vision Profiler (UVP5) (Picheral et al. 2010). The original sampling
methodology and instrument specification followed details described in
Barth and Stone (2022). The UVP was attached to the CTD rosette and
deployed regularly on cruises to the Sargasso Sea from June 2019 -
December 2021. Typical monthly cruises included \textasciitilde13
profiles with average descents to 1200m (Supplemental Figure S1). In
this study, we investigated general trends in DVM by pooling together
casts across multiple cruises. This approach is necessitated by the
small sampling volume of the UVP (1.1L/image) and low abundance of
plankton which requires aggregation of data to resolve trends (see
details in Barth and Stone 2022). While there was minor variation
between cruises (Supplemental Figure S2), this oligotrophic system is
relatively consistent across seasons (Steinberg et al. 2001).
Additionally, every cruise had an approximately equal number of day and
night casts. Profiles were assigned to be day or night based on locally
calculated nautical dawn and nautical dusk times using the R package
\texttt{suncalc\ 0.5.1}.

The UVP records images of large particles (\textgreater600\(\mu\)m
equivalent spherical diameter, ESD). However, living particles are not
reliably identifiable below 900\(\mu\)m (Barth and Stone 2022). All
recorded images were processed using Zooprocess (Gorsky et al. 2010),
which provides several metrics related to size, grey value, and shape
complexity. These features were then used to automatically sort images
using Ecotaxa (Picheral et al. 2017). All images were manually verified
by the same trained taxonomist. In total, 294,913 images were recorded.
Of these, 85.2\% were images of debris or artefacts. The smallest
identified copepod was 0.940mm ESD and the largest was 5.904mm ESD.
Across all casts, copepods were the most common organism, composing
58.7\% of all identified, living particles. In total, there were 4151
individual copepods images.

\hypertarget{morphological-grouping}{%
\subsection{Morphological Grouping}\label{morphological-grouping}}

Zooprocess measures and collects several morphologically relevant
parameters. To create relevant groups of copepods, a dimension reduction
approach was used. Similar methods have been successfully utilized to
provide novel insights to marine snow (Trudnowska et al. 2021;
Szeligowska et al. 2021), copepod dynamics in the Arctic (Vilgrain et
al. 2021), and temporal trends in phytoplankton communities (Sonnet et
al. 2022). First, 18 morphologically relevant parameters were selected
to be included in a Principal Components Analysis (PCA), following
(Vilgrain et al. 2021). Parameters can be described as relating to size
(e.g.~major axis, equivalent spherical diameter {[}ESD{]}), grey
intensity (e.g.~mean grey value at 625nm wavelength light), shape
(e.g.~elongation, symmetry), and shape complexity (e.g.~fractal
dimension). Grey-value intensity specifically can capture a variety of
characteristics related to particle transparency (Gorsky et al. 2010).
Note that the UVP5 utilizes a narrow band pass filter set to 625 nm,
removing the effect of ambient lighting on particle transparency metrics
(Picheral et al. 2010). Feeding these multiple metrics into a
morphospace analysis has several advantages. First, Principle Components
establish the major axes of variability which can aid in interpreting
the relative importance of different traits. Furthermore, in the context
of this study, there are several factors which influence copepod
transparency which are not easily distinguishable in most UVP images. If
only one metric was selected it may only capture one aspect of
transparency, thus by including all factors, we can create a composite
metric. Such approaches have been utilized successfully to infer
characteristics in in-situ imaged marine snow (Trudnowska et al. 2021;
Szeligowska et al. 2021).

The PCA was weighted by the volume sampled in a 1-m depth bin for each
observation. This approach provides a correction for the UVP's variable
descent speed which can cause duplicate imaging of individuals. While
this phenomena has a minor impact on overall results (Barth and Stone
2022), we used the weighted approach to assure that no individual
features were overrepresented. All morphological descriptors were scaled
and centered prior to inclusion in the analysis. The model was
constructed using the R package \texttt{FactoMineR\ 2.7}. Principal
Components (PCs) were deemed to be significant if their eigenvalues were
greater than 1. This approach yielded 4 PCs which described 87.3\% of
the total variation in morphological parameters, with 34.5\% and 26.5\%
in the first two components respectively. The third and fourth PCs were
related to the orientation of the copepod and appendage visibility
respectively. Presumably, this is an artifact of how the copepod was
imaged. Because all axes in a PCA are orthogonal to one another, the
variation captured by PC1 and PC2 are largely spread evenly across the
copepod image variability (PC3 \& PC4). This is a particularly useful
feature as copepod orientation presumably impacts some metrics such as
size and grey-value. Yet, because orientation is largely accounted for
with PC3, by grouping along the first two PCs, variation attributable to
orientation is homogeneous across those axes.

To address our morphology-DVM hypotheses, we constructed discrete
morphological groups based on the first two principal components. Groups
along each of the principal components were defined as low (below 25th
percentile), mid (25th-75th percentile) and high (greater than 75th
percentile). To address the size-dependent hypothesis (H1), groups were
assigned as low, mid, or high along PC1. Then to assess if
color/transparency was a secondary factor (H2), within each PC1 group,
PC2 groups were constructed as low, mid, or high. In total, this created
9 groups (e.g.~Low PC1-Low PC2, Low P1-mid PC2, etc).

\hypertarget{copepod-vertical-structure-dvm}{%
\subsection{Copepod vertical structure \&
DVM}\label{copepod-vertical-structure-dvm}}

\hypertarget{vertical-distribution-of-copepods}{%
\subsubsection{Vertical distribution of
copepods}\label{vertical-distribution-of-copepods}}

Copepods in this system are well documented to undergo DVM (Steinberg et
al. 2000; Schnetzer and Steinberg 2002; Maas et al. 2018). However,
there have not been direct measurements of DVM with in-situ imaging
data. First, to assess which portion of the water column copepods were
utilizing for DVM, we visualized the average vertical structure. The
concentrations of each morphological group (based on PC1 and PC2) were
calculated in 20m depth bins for each UVP profile. Profiles were
designated as either day or night. Then across all day/night profiles,
the mean concentration was calculated for each 20m depth bin.

\hypertarget{weighted-mean-depth-variability}{%
\subsubsection{Weighted mean depth
variability}\label{weighted-mean-depth-variability}}

Weighted mean depth (WMD) is a common metric to describe vertical
structure and DVM in zooplankton (Ohman et al. 2002; Ohman and Romagnan
2016; Aarflot et al. 2019). However, with in-situ imagery and our
particular dataset, this approach presents a few challenges. WMD cannot
be calculated individually for each profile then averaged because many
profiles in this study had different descent depths. Additionally, the
small and uneven sampling volume of the UVP can make single casts too
variable to reliably resolve abundance. Yet, understanding variation
around the WMD is necessary to compare DVM strength across groups. Here,
we introduce a depth-bin constrained bootstrap approach to define WMD
with a 95\% confidence interval. To do this, the concentration of each
group, was calculated in 20m depth bins for each profile. Then all
profiles were `pooled', separately for day/night. This provides a
distribution of concentrations in each depth-bin. Pooling across
multiple seasons was necessary to have sufficient data, however it does
introduce additionally variability. Due to the different descent speeds
and depth of profiles, there are more observations of surface depth
bins. Thus, traditional bootstrapping would bias estimate toward the
surface as resampling would be more likely to draw a more-frequently
observed surface bin. To avoid this, bootstrap samples were
``bin-constrained'' such that for each iteration, a random observation
was drawn within each depth bin, then replaced for the next iteration. A
maximum depth was set to 600m based on qualitative observations of
vertical profiles. This approach effectively created a random profile by
resampling a concentration, \(conc^*\), from each depth bin, \(d\). For
each iteration, the random constructed profile then was used to
calculate a bootstrapped weighted mean depth, \(WMD^*\). This was done
for each morphological group, \(g\), at each time of day, \(t\)
(day/night).

\[WMD^*_{g,t} = \sum_{i}^{N = 30}{\frac{d_i(conc^*_{i,g,t})}{\sum_i^{N = 30}{conc^*_{i,g,t}}}}\]

The distribution of \(WMD^*_{g,t}\) then was used to calculate a
bootstrapped mean and 95\% confidence interval. The width of the
confidence interval then is influenced both by the spread of copepods
through the water column and the amount of data available to confidently
support their estimates. Thus, this resampling approach is conservative
in identifying a significant trend. The conservative approach is
desirable given both its robustness to UVP sampling variability and the
need to pool casts as described above. To assess a DVM pattern, the 95\%
CIs can be compared between times of day and morphological groups. We
define a clear DVM signal (e.g.~significant day/night difference) as
when there is no overlap between between the 95\% WMD CIs between
nighttime and daytime groups. If a clear signal was observed, the DVM
magnitude can be measured by comparing the mean \(WMD^*\)s.

With PC1 to assess the size-based hypothesis (H1), the WMD was compared
between the three PC1-groups by percentile level. Then to assess the
effect of transparency (H2) the WMD was compared between PC2-groups
within each PC1-grouping.

\hypertarget{results}{%
\section{Results}\label{results}}

\hypertarget{morphological-groups}{%
\subsection{Morphological Groups}\label{morphological-groups}}

The PCA revealed four major axes of variability (Figure 1). The first
axis (PC1, 34.23\% of variability) was largely explained by increasing
values related to size, such as perimeter (loading score = 0.927) and
feret diameter (maximum distance between parallel planes around an
object) (loading score = 0.910). The second axis (PC2, 27.24\% of
variability) can be interpreted as a gradient of transparent to dark
individuals. PC2 was largely anticorrelated with mean grey value (higher
values indicate a more transparent individual) (loading score = -0.920).
As noted in the methods, PC3 and PC4 were both related to the
orientation of the copepod and the appendage visibility respectively
(Supplemental Figure 3).

\begin{figure}

{\centering \includegraphics{../media/figure_01.pdf}

}

\caption{First two principal components of the morphospace. Proportion
of variance explained by the two axis is 61.1\%. Each point represents
an individual copepod. The color and transparency of each point
corressponds to the morphological groups based on pecentile along each
axis. Along PC1, grey corresponds to the low-group (\textless25th
percentile), orange to the mid group (25th-75th percentiles), and blue
to the high-group (75th percentile). Along PC2, low, mid, and high
groups are distinguished by increasing opacity. Marginal distribution
display the proportion of observations in each group. Representative
vignettes of copepods are shown in the corners corresponding to their
place in the morphospace. 4mm scale bar in the bottom right is shown for
the vignettes.}

\end{figure}

The morphological groupings were assigned along PC1 as low, mid and
high. Then along PC2, groups were assigned within each PC1-group (Figure
1). To confirm the morphospace grouping resulted in ecologically
relevant categories, the morphological groups were compared against
known copepod metrics. Across all PC1-groups, there was a clear
difference in feret diameter. The median feret diameter of the low group
was 1.97mm. The median feret diameter of the mid and high groups were
2.84mm and 4.83mm, respectively (Figure 2A). All groups were
significantly different from one another (Dunn Kruskall-Wallace test, p
\textless{} 0.001). PC2 groups as a whole were also significantly
different from one another (Dunn Krustall-wallace test, p \textless{}
0.001). However, within each PC2-group, there was a clear tendency for
larger copepods (high PC1 group) to be more transparent (Figure 2B).

\begin{figure}

{\centering \includegraphics{../media/figure_02.pdf}

}

\caption{Comparison of morphological groups to relevant parameters.
Groups were constructed along principal components with low as below
25th percentile, mid as 25th-50th percentile, and high as above 75th
percentile. (A) PC1 groups are significantly different along feret
diameter and display a clear trend for size. (B) PC2 groups are
significantly different in terms of mean grey value. Note that a low
mean grey value indicates a darker copepod.}

\end{figure}

\hypertarget{vertical-profiles-of-morphological-groups}{%
\subsection{Vertical Profiles of Morphological
Groups}\label{vertical-profiles-of-morphological-groups}}

For all groups, the 20m-binned profiles show a notable structure. While
copepods were observed throughout the mesopelagic (Supplemental Figure
4), the majority of day/night differences were observed above 600m
(Figure 3). For most morphological groups, there was a peak in nighttime
concentration in the lower epipelagic (50m-200m). Similarly, there was a
decrease in average daytime concentration over the same region. This
pattern is particularly apparent for the groups which are mid and high
on both PCs (Figure 3B, C, E, F). Across all groups, both average
daytime and nighttime concentration were low in the upper mesopelagic
(200m-300m). Then, there was a peak in average daytime concentration in
the depth bins in the mid-mesopelagic (400m-600m).

\begin{figure}

{\centering \includegraphics{../media/figure_03.pdf}

}

\caption{Average vertical profile of different copepod morphological
groups. Bars display average concentration in a 20m depth bin. On each
panel, left-side bars (tan) correspond to daytime while right-side
(teal) bars correspond to nighttime. Standard deviation is shown for
each 20m depth bin. Each panel corresponds to a morphological group
along PC1 (size axis) and PC2 (transparency axis). (A) low PC1, high
PC2; (B) mid PC1, high PC2; (C) high PC1, high PC2; (D) low PC1, mid
PC2; (E) mid PC1, mid PC2; (F) high PC1, mid PC2; (G) low PC1, low PC2;
(H) mid PC1, low PC2; (I) high PC1, low PC2}

\end{figure}

\hypertarget{weighted-mean-depth-analysis}{%
\subsection{Weighted mean depth
analysis}\label{weighted-mean-depth-analysis}}

The bin-constrained bootstrap approach provided a direct method to
compare DVM between groups. Size (PC1) had a clear effect on DVM
magnitude. First, for all PC1 groups, daytime WMD 95\% bootstrapped
confidence intervals (95\% CIs) were deeper and non-overlapping with the
nighttime 95\% CIs (Figure 4). This indicates a clear DVM pattern.
However, the differences in day and night CIs varied between
morphological groups. All PC1 groups had a similar, overlapping
nighttime 95\% CI in the lower epipelagic (\textasciitilde145m -
\textasciitilde200m). However, there was a clear difference in the depth
of the daytime 95\% CIs. The small (low PC1) group had the shallowest
95\% CI (235.2m-296.0m). The mid PC1 group's daytime 95\% CI was
slightly deeper (309.0m-347.3m). The large (high PC1) group daytime 95\%
CI was even lower (352.3m-405.0m).

\begin{figure}

{\centering \includegraphics{../media/figure_04.pdf}

}

\caption{Mean bootstrapped weightd mean depth and 95\% confidence
intervals for copepods of different morphological groups. Low, mid, and
high groups correspond to the different percentiles along PC1 from the
morphospace. PC1 largely is explained by size metrics, with higher
scores indicating a larger copepod.}

\end{figure}

When considering the influence of transparency (PC2) on DVM magnitude,
we compared PC2 groups within their PC1 grouping. This approach was
warranted because of the tendency for size to have a slight effect on
transparency (Figure 2). At this level of comparison, there were several
notable trends. For the smaller copepods (low PC1), once the data were
split into PC2 groups, the wider 95\% CIs indicate little to no DVM
signal. Generally, the daytime 95\% CIs and nighttime 95\% CIs are
overlapping or near-overlapping (Figure 5A). With mid-sized copepods,
there was a clear DVM signal. However, all PC2 groups appeared to have a
similar DVM magnitude with each group's daytime 95\% CIs overlapping
with each other (Figure 5B). There was a difference in DVM magnitude
across PC2 groups within the largest copepods. The more transparent
copepods (low PC2 group) showed no DVM signal, with a shallow daytime
WMD. However, the darker copepods (mid and high PC2 groups) had deeper
daytime WMDs (Figure 5c).

\begin{figure}

{\centering \includegraphics{../media/figure_05.pdf}

}

\caption{Mean bootstrapped weighted mean depth and 95\% confidence
intervals shown by copepod morphological groups along PC2
(transparency). Each panel represents a different size group of copepods
(PC1 groups).}

\end{figure}

\hypertarget{discussion}{%
\section{Discussion}\label{discussion}}

\hypertarget{copepod-morphospace-quantifying-dvm}{%
\subsection{Copepod morphospace \& quantifying
DVM}\label{copepod-morphospace-quantifying-dvm}}

In this study, we built on methods for describing morphospaces from
similar in-situ imaging studies (Vilgrain et al. 2021; Trudnowska et al.
2021; Sonnet et al. 2022). The PCA-defined morphospace with the present
data aligns well with the prior applications. Interestingly, the
proportion of morphological variation explained by each axis in the
morphospace defined on Arctic copepods by Vilgrain et al. (2021) is
extremely similar to the morphospace axes in this study. This similarity
is striking considering the vastly different copepod community
compositions between the Arctic ocean and subtropical gyres (Soviadan et
al. 2022). The similarity of morphospaces could also be an artifact of
the similarity of input data. Given the UVP has a limited range of
observable size classes (Picheral et al. 2010), only copepods above a
certain size were fed into both PCAs. Alternatively, the similarity of
studies suggest that copepod morphological variation might be well
described by these two primary axes. (Sonnet et al. 2022) used
phytoplankton images to investigate how a morphospace could be used to
evaluate community composition changes over time. Comparisons of copepod
morphospaces across temporal and spatial scales may offer a useful
metric for answering biogeographic and ecological questions.

While the UVP provides some methodological challenges to quantifying
DVM, the pattern of DVM described in this study is consistent with the
commonly observed nocturnal DVM pattern (Bianchi and Mislan 2016;
Bandara et al. 2021). The average vertical profiles display a clear
day/night difference (Figure 3). However, in each 20m depth bin there
was large variation, often exceeding the average concentration. This
large variation was expected. There can be considerable variation
between UVP estimates of zooplankton abundance (Barth and Stone 2022).
Additionally, in this study we pooled casts across multiple seasons.
Variability in copepod DVM has been described across seasons (Whitmore
and Ohman 2021). However, in the Sargasso Sea, while there is seasonal
variation in DVM biomass (Behrenfeld 2014), there is no record of
variation in DVM magnitude. Other studies describing DVM in the region
have also pooled across seasons using net data (Ivory et al. 2019).
Thus, while pooling across seasons may have introduced some variability
in our WMD estimates, the DVM signal was still well described by the
UVP. Previous studies using in-situ imaging have also noted a signal of
DVM with copepods (Pan et al. 2018; Whitmore and Ohman 2021). Yet due to
small and uneven sampling, it can be a challenge to quantify DVM using
in-situ imaging. As presented in this paper, bin-constrained
bootstrapping offers a robust method to quantify WMD and investigate DVM
hypotheses.

\hypertarget{morphological-variation-in-dvm}{%
\subsection{Morphological variation in
DVM}\label{morphological-variation-in-dvm}}

The results presented in this study provide new perspective on how
traits influence DVM patterns. Consistent with the size-based hypothesis
(H1), we documented a clear effect in which larger copepods migrated
further. This finding is consistent with several studies which have
documented a size-dependent relationship for copepod DVM (Ohman and
Romagnan 2016; Aarflot et al. 2019; Pinti et al. 2019). Ohman and
Romagnan (2016) noted that moderate-size copepods had the largest
migrations. While this may seem contradictory to the present study, the
difference between study systems needs be taken into account. The
copepods described in the large (high PC1) group had a mean feret
diameter of nearly 5mm. Conversely, in Ohman and Romagnan (2016)'s study
the ``moderate'' copepods ranged from 4mm-6mm.

The transparency-based hypothesis (H2) was only supported by patterns
within the large copepod group. The large but more transparent copepods
(low PC2, high PC1) did not have a detectable DVM signal. Yet the darker
copepods (mid and high PC2) had a large DVM signal. It should be noted
that the grey-value recorded by the UVP may be indicative of many
features which influence copepod pigmentation, including pigmentation,
egg-sacs and gut contents (Vilgrain et al. 2021). Thus, while our
observation is consistent with both the predator-avoidance and the
hunger-satiation hypotheses, we cannot distinguish exactly why the
large, more transparent copepods do not migrate. One possibility is that
these copepods have empty gut contents, and thus are less transparent
and motivated to feed near the surface. However it is aslo plausible
that the difference in transparency is driven by taxonomic differences
in pigmentation. UVP images of copepods are generally unidentifiable to
higher taxonomic resolution. However, it is likely that the majority of
copepod images were Calanoida, which are consistently reported as the
dominant copepod group in the Sargasso Sea (Deevey and Brooks 1977;
Ivory et al. 2019; Blanco-Bercial 2020). Additionally, a long-term
analysis of net-collected data reported only Calanoida to show a
significant DVM signal (Ivory et al. 2019). However, within this group,
there is extreme diversity (Deevey and Brooks 1977; Blanco-Bercial
2020). In a metabarcoding study of the epipelagic mesozooplankton
community, Blanco-Bercial (2020) reported \emph{Pleuromamma spp.,}
Euchaetidae, and Eucalanoidea to show higher nighttime relative
abundance. Alternatively, Calanidae were described to occupy the surface
waters at daytime. Thus, while the present study cannot make direct
conclusions as to taxonomic variation, it is likely a driving factor in
DVM variability across the observed morphological groups.

Hays (2003) described that copepod pigmentation could explain increased
DVM with small (\textless1mm) copepods. Thus it was surprising that
there was no effect of transparency on DVM magnitude in the smaller
morphology group. One possibility is that the small, transparent
copepods were not well sampled by the UVP (Figure 2). Due to the
conservative nature of the bootstrapping WMD approach utilized in this
study, sampling deficits would broaden the 95\% WMD CI's, reducing the
ability to resolve trends. However, there is no observable trend to
suggest the lack of DVM signal is simply a methodological artifact.
Alternatively, there are several possible explanations for why there is
no effect of transparency for the smaller copepod morphological groups.
First, it should be noted that while both the small-sized (low PC1) and
mid-sized (mid-PC1) group showed no variation in DVM patterns across
PC2, the mid-sized group consistently showed a strong DVM signal while
the small-sized group did not. Amongst the small-sized group, there was
little observable DVM across all transparency groupings. It may be that
these copepods have a low visual-predator risk regardless of their
transparency level. Ohman and Romagnan (2016)'s study in the California
Current observed that the smallest copepods (\textless1.5mm) displayed
no DVM signal. While these copepods may have reduced predator risk, it
may also be that they simply are weaker swimmers and cannot reasonably
migrate as large of distances as bigger copepods can.

The mid-sized copepods display a clear DVM signal across all
transparency groupings. This suggests that mid-sized copepods do not
relieve their predation risk through increased transparency. This is
counterintuitive to the observation that the large, transparent copepods
have a reduced DVM signal. Additionally the transparent, mid-sized
copepods migrating while the transparent, large ones do not, directly
contradicts the predator-avoidance hypothesis. It is worth noting that
DVM behavior can also vary greatly across species. Within migrating
nekton, there has been mixed support described for the hunger-satiation
hypothesis, depending on taxa (Bos et al. 2021). Thus again it may be
taxonomic variation which can explain deviations in expected DVM
patterns. Additionally, other mechanisms influencing DVM, asides from
top-down factors may be at play. Williamson et al. (2011) provides
support for the transparency-regulator hypothesis of DVM, which suggests
both top-down factors and environmental factors such as UV-radiation
influence DVM behavior. In the Sargasso Sea, there is extreme water
clarity which would suggest UV-radiation may play a role in DVM
behavior. However, none of our findings suggest that more pigmented
copepods migrate less than transparent ones, regardless of size. Yet,
across all copepod morphological groups, abundances were highest in the
mid-to-lower mesopelagic and low in the surface layers (Figure 2). Thus
the copepods imaged in this study are likely already at layers below
where UV damage is a major factor. Furthermore, the deep chlorophyll-a
maximum regularly extended into low epipelagic (Supplemental Figure S2),
providing sufficient food where UV irradiance is low. Nonetheless, while
UV may not be a primary factor, the notion that there are multiple
factors influencing DVM should be considered.

\hypertarget{conclusion}{%
\subsection{Conclusion}\label{conclusion}}

Overall, our results reveal a complex dynamic between copepod traits and
DVM behavior. This study provided new insight into the DVM dynamics in
oligotrophic gyres. While many studies have established size as a major
trait influencing DVM, investigations into other traits are more
limited. Here, we support the prevailing notion that size has large
consequences for DVM behavior. We also show that transparency has an
effect on DVM for some size groups. However, determining exactly which
drivers determine why copepods with different traits undergo diel
vertical migration remains elusive. While these findings are largely
consistent with the predator-avoidance hypothesis and prevailing DVM
theory, they highlight the need for more detailed analyses. As plankton
in-situ imaging tools are used more commonly by oceanographers, larger
datasets will facilitate new investigations. Additionally, improved
resolution of sampling tools may better determine taxonomic variation.
Collaborations between oceanographers, plankton ecologists, and visual
ecologists may better resolve how traits influence trade-offs in DVM
behavior. Understanding these dynamics will be critical to predicting
changes with a changing ocean.

\hypertarget{acknowledgements}{%
\section{Acknowledgements}\label{acknowledgements}}

Field work for this project was supported by the Bermuda Atlantic Time
Series Study through NSF OCE 1756105 \& NSF OCE 1756312. We would also
like to thank the BATS research technicians, marine technicians, and
crew of the R/V Atlantic Explorer. Dr.~Ryan Rykaczewski assisted with
the initial set-up of the UVP. Dr.~Leo Blanco-Bercial and Dr.~Amy Maas
both valuable insight and guidance on the analysis.

\hypertarget{data-availability-statement}{%
\section{\texorpdfstring{\emph{Data availability
statement}}{Data availability statement}}\label{data-availability-statement}}

All data used in this project are hosted on Ecopart
(\url{https://ecopart.obs-vlfr.fr/}). Data in its raw form can be
accessed from their portal. However, all summary and intermediate data
products, as well as code, are publicly available on GitHub
(\url{https://github.com/TheAlexBarth/DVM_Migration-Morphology}).
Intermediate data products are formatted as R Data Structure objects,
other formats are available on request.

\hypertarget{references}{%
\subsection{References}\label{references}}

\hypertarget{refs}{}
\begin{CSLReferences}{1}{0}
\leavevmode\vadjust pre{\hypertarget{ref-aarflot2019}{}}%
Aarflot, J. M., D. L. Aksnes, A. F. Opdal, H. R. Skjoldal, and Ø.
Fiksen. 2019. Caught in broad daylight: Topographic constraints of
zooplankton depth distributions. Limnology and Oceanography \textbf{64}:
849--859.
doi:\href{https://doi.org/10.1002/lno.11079}{10.1002/lno.11079}

\leavevmode\vadjust pre{\hypertarget{ref-aksnes1997}{}}%
Aksnes, D. L., and A. C. W. Utne. 1997. A revised model of visual range
in fish. Sarsia \textbf{82}: 137--147.
doi:\href{https://doi.org/10.1080/00364827.1997.10413647}{10.1080/00364827.1997.10413647}

\leavevmode\vadjust pre{\hypertarget{ref-archibald2019}{}}%
Archibald, K. M., D. A. Siegel, and S. C. Doney. 2019. Modeling the
Impact of Zooplankton Diel Vertical Migration on the Carbon Export Flux
of the Biological Pump. Global Biogeochemical Cycles \textbf{33}:
181--199.
doi:\href{https://doi.org/10.1029/2018GB005983}{10.1029/2018GB005983}

\leavevmode\vadjust pre{\hypertarget{ref-atkinson1992}{}}%
Atkinson, A., P. Ward, R. Williams, and S. A. Poulet. 1992. Diel
vertical migration and feeding of copepods at an oceanic site near South
Georgia. Marine Biology \textbf{113}: 583--593.
doi:\href{https://doi.org/10.1007/BF00349702}{10.1007/BF00349702}

\leavevmode\vadjust pre{\hypertarget{ref-bandara2019}{}}%
Bandara, K., Ø. Varpe, R. Ji, and K. Eiane. 2019. Artificial evolution
of behavioral and life history strategies of high-latitude copepods in
response to bottom-up and top-down selection pressures. Progress in
Oceanography \textbf{173}: 134--164.
doi:\href{https://doi.org/10.1016/j.pocean.2019.02.006}{10.1016/j.pocean.2019.02.006}

\leavevmode\vadjust pre{\hypertarget{ref-bandara2021}{}}%
Bandara, K., Ø. Varpe, L. Wijewardene, V. Tverberg, and K. Eiane. 2021.
Two hundred years of zooplankton vertical migration research. Biological
Reviews \textbf{96}: 1547--1589.
doi:\href{https://doi.org/10.1111/brv.12715}{10.1111/brv.12715}

\leavevmode\vadjust pre{\hypertarget{ref-barth2022}{}}%
Barth, A., and J. Stone. 2022.
\href{https://www.frontiersin.org/articles/10.3389/fmars.2022.898057}{Comparison
of an in situ imaging device and net-based method to study
mesozooplankton communities in an oligotrophic system}. Frontiers in
Marine Science \textbf{9}.

\leavevmode\vadjust pre{\hypertarget{ref-behrenfeld2014}{}}%
Behrenfeld, M. J. 2014. Climate-mediated dance of the plankton. Nature
Climate Change \textbf{4}: 880--887.
doi:\href{https://doi.org/10.1038/nclimate2349}{10.1038/nclimate2349}

\leavevmode\vadjust pre{\hypertarget{ref-behrenfeld2019}{}}%
Behrenfeld, M. J., P. Gaube, A. Della Penna, and others. 2019. Global
satellite-observed daily vertical migrations of ocean animals. Nature
\textbf{576}: 257--261.
doi:\href{https://doi.org/10.1038/s41586-019-1796-9}{10.1038/s41586-019-1796-9}

\leavevmode\vadjust pre{\hypertarget{ref-bianchi2016}{}}%
Bianchi, D., and K. a. S. Mislan. 2016. Global patterns of diel vertical
migration times and velocities from acoustic data. Limnology and
Oceanography \textbf{61}: 353--364.
doi:\href{https://doi.org/10.1002/lno.10219}{10.1002/lno.10219}

\leavevmode\vadjust pre{\hypertarget{ref-blanco-bercial2020}{}}%
Blanco-Bercial, L. 2020.
\href{https://www.frontiersin.org/articles/10.3389/fmars.2020.00173}{Metabarcoding
analyses and seasonality of the zooplankton community at BATS}.
Frontiers in Marine Science \textbf{7}.

\leavevmode\vadjust pre{\hypertarget{ref-bos2021}{}}%
Bos, R. P., T. T. Sutton, and T. M. Frank. 2021.
\href{https://www.frontiersin.org/articles/10.3389/fmars.2021.607228}{State
of satiation partially regulates the dynamics of vertical migration}.
Frontiers in Marine Science \textbf{8}.

\leavevmode\vadjust pre{\hypertarget{ref-brierley2014}{}}%
Brierley, A. S. 2014. Diel vertical migration. Current Biology
\textbf{24}: R1074--R1076.
doi:\href{https://doi.org/10.1016/j.cub.2014.08.054}{10.1016/j.cub.2014.08.054}

\leavevmode\vadjust pre{\hypertarget{ref-cohen2009}{}}%
Cohen, J. H., and R. B. Forward Jr. 2009.
\href{https://www.vliz.be/en/imis}{Zooplankton diel vertical migration:
a review of proximate control}, p. 77--110. \emph{In}.

\leavevmode\vadjust pre{\hypertarget{ref-deevey1977}{}}%
Deevey, G. B., and A. L. Brooks. 1977. Copepods of the sargasso sea off
bermuda: Species composition, and vertical and seasonal distribution
between the surface and 2000 m. Bulletin of Marine Science \textbf{27}:
256--291.

\leavevmode\vadjust pre{\hypertarget{ref-enright1977}{}}%
Enright, J. T. 1977. Diurnal vertical migration: Adaptive significance
and timing. Part 1. Selective advantage: A metabolic model1. Limnology
and Oceanography \textbf{22}: 856--872.
doi:\href{https://doi.org/10.4319/lo.1977.22.5.0856}{10.4319/lo.1977.22.5.0856}

\leavevmode\vadjust pre{\hypertarget{ref-ewald1912}{}}%
Ewald, W. F. 1912. On artificial modification of light reactions and the
influence of electrolytes on phototaxis. Journal of Experimental Zoology
\textbf{13}: 591--612.
doi:\href{https://doi.org/10.1002/jez.1400130405}{10.1002/jez.1400130405}

\leavevmode\vadjust pre{\hypertarget{ref-gorsky2010}{}}%
Gorsky, G., M. D. Ohman, M. Picheral, and others. 2010. Digital
zooplankton image analysis using the ZooScan integrated system. Journal
of Plankton Research \textbf{32}: 285--303.
doi:\href{https://doi.org/10.1093/plankt/fbp124}{10.1093/plankt/fbp124}

\leavevmode\vadjust pre{\hypertarget{ref-hays2003}{}}%
Hays, G. C. 2003. \href{https://doi.org/10.1007/978-94-017-2276-6_18}{A
review of the adaptive significance and ecosystem consequences of
zooplankton diel vertical migrations}. Springer Netherlands. 163--170.

\leavevmode\vadjust pre{\hypertarget{ref-hays1994}{}}%
Hays, G. C., C. A. Proctor, A. W. G. John, and A. J. Warner. 1994.
Interspecific differences in the diel vertical migration of marine
copepods: The implications of size, color, and morphology. Limnology and
Oceanography \textbf{39}: 1621--1629.
doi:\href{https://doi.org/10.4319/lo.1994.39.7.1621}{10.4319/lo.1994.39.7.1621}

\leavevmode\vadjust pre{\hypertarget{ref-ivory2019}{}}%
Ivory, J. A., D. K. Steinberg, and R. J. Latour. 2019. Diel, seasonal,
and interannual patterns in mesozooplankton abundance in the sargasso
sea. ICES Journal of Marine Science \textbf{76}: 217--231.
doi:\href{https://doi.org/10.1093/icesjms/fsy117}{10.1093/icesjms/fsy117}

\leavevmode\vadjust pre{\hypertarget{ref-kelly2019}{}}%
Kelly, T. B., P. C. Davison, R. Goericke, M. R. Landry, M. D. Ohman, and
M. R. Stukel. 2019.
\href{https://www.frontiersin.org/articles/10.3389/fmars.2019.00508}{The
importance of mesozooplankton diel vertical migration for sustaining a
mesopelagic food web}. Frontiers in Marine Science \textbf{6}.

\leavevmode\vadjust pre{\hypertarget{ref-kessler2008}{}}%
Kessler, K., R. S. Lockwood, C. E. Williamson, and J. E. Saros. 2008.
Vertical distribution of zooplankton in subalpine and alpine lakes:
Ultraviolet radiation, fish predation, and the transparency-gradient
hypothesis. Limnology and Oceanography \textbf{53}: 2374--2382.
doi:\href{https://doi.org/10.4319/lo.2008.53.6.2374}{10.4319/lo.2008.53.6.2374}

\leavevmode\vadjust pre{\hypertarget{ref-lampert1989}{}}%
Lampert, W. 1989. The adaptive significance of diel vertical migration
of zooplankton. Functional Ecology \textbf{3}: 21--27.
doi:\href{https://doi.org/10.2307/2389671}{10.2307/2389671}

\leavevmode\vadjust pre{\hypertarget{ref-liu2022}{}}%
Liu, Y., J. Guo, Y. Xue, C. Sangmanee, H. Wang, C. Zhao, S.
Khokiattiwong, and W. Yu. 2022. Seasonal variation in diel vertical
migration of zooplankton and micronekton in the andaman sea observed by
a moored ADCP. Deep Sea Research Part I: Oceanographic Research Papers
\textbf{179}: 103663.
doi:\href{https://doi.org/10.1016/j.dsr.2021.103663}{10.1016/j.dsr.2021.103663}

\leavevmode\vadjust pre{\hypertarget{ref-maas2018}{}}%
Maas, A. E., L. Blanco-Bercial, A. Lo, A. M. Tarrant, and E.
Timmins-Schiffman. 2018. Variations in copepod proteome and respiration
rate in association with diel vertical migration and circadian cycle.
The Biological Bulletin \textbf{235}: 30--42.
doi:\href{https://doi.org/10.1086/699219}{10.1086/699219}

\leavevmode\vadjust pre{\hypertarget{ref-mclaren1963}{}}%
McLaren, I. A. 1963. Effects of temperature on growth of zooplankton,
and the adaptive value of vertical migration. Journal of the Fisheries
Research Board of Canada \textbf{20}: 685--727.
doi:\href{https://doi.org/10.1139/f63-046}{10.1139/f63-046}

\leavevmode\vadjust pre{\hypertarget{ref-mclaren1974}{}}%
McLaren, I. A. 1974.
\href{https://www.jstor.org/stable/2459738}{Demographic strategy of
vertical migration by a marine copepod}. The American Naturalist
\textbf{108}: 91--102.

\leavevmode\vadjust pre{\hypertarget{ref-ohman1990}{}}%
Ohman, M. D. 1990. The Demographic Benefits of Diel Vertical Migration
by Zooplankton. Ecological Monographs \textbf{60}: 257--281.
doi:\href{https://doi.org/10.2307/1943058}{10.2307/1943058}

\leavevmode\vadjust pre{\hypertarget{ref-ohman2019b}{}}%
Ohman, M. D. 2019. A sea of tentacles: optically discernible traits
resolved from planktonic organisms in situ H. Browman {[}ed.{]}. ICES
Journal of Marine Science \textbf{76}: 1959--1972.
doi:\href{https://doi.org/10.1093/icesjms/fsz184}{10.1093/icesjms/fsz184}

\leavevmode\vadjust pre{\hypertarget{ref-ohman2016}{}}%
Ohman, M. D., and J.-B. Romagnan. 2016. Nonlinear effects of body size
and optical attenuation on Diel Vertical Migration by zooplankton.
Limnology and Oceanography \textbf{61}: 765--770.
doi:\href{https://doi.org/10.1002/lno.10251}{10.1002/lno.10251}

\leavevmode\vadjust pre{\hypertarget{ref-ohman2002}{}}%
Ohman, M. D., J. A. Runge, E. G. Durbin, D. B. Field, and B. Niehoff.
2002. On birth and death in the sea. Hydrobiologia \textbf{480}: 55--68.
doi:\href{https://doi.org/10.1023/A:1021228900786}{10.1023/A:1021228900786}

\leavevmode\vadjust pre{\hypertarget{ref-pan2018}{}}%
Pan, J., F. Cheng, and F. Yu. 2018.
\href{http://nopr.niscpr.res.in/handle/123456789/44619}{The diel
vertical migration of zooplankton in the hypoxia area observed by video
plankton recorder}. IJMS Vol.47(07) {[}July 2018{]}.

\leavevmode\vadjust pre{\hypertarget{ref-pearrejr.2003}{}}%
Pearre, S. 2003. Eat and run? The hunger/satiation hypothesis in
vertical migration: history, evidence and consequences. Biological
Reviews \textbf{78}: 1--79.
doi:\href{https://doi.org/10.1017/S146479310200595X}{10.1017/S146479310200595X}

\leavevmode\vadjust pre{\hypertarget{ref-picheral}{}}%
Picheral, M., S. Colin, and J.-O. Irisson. 2017.
\href{http://ecotaxa.obs-vlfr.fr}{EcoTaxa, a tool for the taxonomic
classification of images.}

\leavevmode\vadjust pre{\hypertarget{ref-picheral2010}{}}%
Picheral, M., L. Guidi, L. Stemmann, D. M. Karl, G. Iddaoud, and G.
Gorsky. 2010. The Underwater Vision Profiler 5: An advanced instrument
for high spatial resolution studies of particle size spectra and
zooplankton. Limnology and Oceanography: Methods \textbf{8}: 462--473.
doi:\href{https://doi.org/10.4319/lom.2010.8.462}{10.4319/lom.2010.8.462}

\leavevmode\vadjust pre{\hypertarget{ref-pinti2019}{}}%
Pinti, J., T. Kiørboe, U. H. Thygesen, and A. W. Visser. 2019. Trophic
interactions drive the emergence of diel vertical migration patterns: A
game-theoretic model of copepod communities. Proceedings of the Royal
Society B: Biological Sciences \textbf{286}: 20191645.
doi:\href{https://doi.org/10.1098/rspb.2019.1645}{10.1098/rspb.2019.1645}

\leavevmode\vadjust pre{\hypertarget{ref-ringelberg2009}{}}%
Ringelberg, J. 2009. Diel Vertical Migration of Zooplankton in Lakes and
Oceans: causal explanations and adaptive significances, Springer Science
\& Business Media.

\leavevmode\vadjust pre{\hypertarget{ref-robison2020}{}}%
Robison, B. H., R. E. Sherlock, K. R. Reisenbichler, and P. R. McGill.
2020.
\href{https://www.frontiersin.org/articles/10.3389/fmars.2020.00064}{Running
the gauntlet: Assessing the threats to vertical migrators}. Frontiers in
Marine Science \textbf{7}.

\leavevmode\vadjust pre{\hypertarget{ref-schnetzer2002}{}}%
Schnetzer, A., and D. K. Steinberg. 2002. Active transport of
particulate organic carbon and nitrogen by vertically migrating
zooplankton in the Sargasso Sea. Marine Ecology Progress Series
\textbf{234}: 71--84.
doi:\href{https://doi.org/10.3354/meps234071}{10.3354/meps234071}

\leavevmode\vadjust pre{\hypertarget{ref-siegel2023}{}}%
Siegel, D. A., T. DeVries, I. Cetinić, and K. M. Bisson. 2023.
Quantifying the ocean's biological pump and its carbon cycle impacts on
global scales. Annual Review of Marine Science \textbf{15}: null.
doi:\href{https://doi.org/10.1146/annurev-marine-040722-115226}{10.1146/annurev-marine-040722-115226}

\leavevmode\vadjust pre{\hypertarget{ref-sonnet2022}{}}%
Sonnet, V., L. Guidi, C. B. Mouw, G. Puggioni, and S.-D. Ayata. 2022.
Length, width, shape regularity, and chain structure: time series
analysis of phytoplankton morphology from imagery. Limnology and
Oceanography \textbf{67}: 1850--1864.
doi:\href{https://doi.org/10.1002/lno.12171}{10.1002/lno.12171}

\leavevmode\vadjust pre{\hypertarget{ref-soviadan2022}{}}%
Soviadan, Y. D., F. Benedetti, M. C. Brandão, and others. 2022. Patterns
of mesozooplankton community composition and vertical fluxes in the
global ocean. Progress in Oceanography \textbf{200}: 102717.
doi:\href{https://doi.org/10.1016/j.pocean.2021.102717}{10.1016/j.pocean.2021.102717}

\leavevmode\vadjust pre{\hypertarget{ref-steinberg2000}{}}%
Steinberg, D. K., C. A. Carlson, N. R. Bates, S. A. Goldthwait, L. P.
Madin, and A. F. Michaels. 2000. Zooplankton vertical migration and the
active transport of dissolved organic and inorganic carbon in the
Sargasso Sea. Deep Sea Research Part I: Oceanographic Research Papers
\textbf{47}: 137--158.
doi:\href{https://doi.org/10.1016/S0967-0637(99)00052-7}{10.1016/S0967-0637(99)00052-7}

\leavevmode\vadjust pre{\hypertarget{ref-steinberg2001}{}}%
Steinberg, D. K., C. A. Carlson, N. R. Bates, R. J. Johnson, A. F.
Michaels, and A. H. Knap. 2001. Overview of the US JGOFS Bermuda
Atlantic Time-series Study (BATS): a decade-scale look at ocean biology
and biogeochemistry. Deep Sea Research Part II: Topical Studies in
Oceanography \textbf{48}: 1405--1447.
doi:\href{https://doi.org/10.1016/S0967-0645(00)00148-X}{10.1016/S0967-0645(00)00148-X}

\leavevmode\vadjust pre{\hypertarget{ref-steinberg2017}{}}%
Steinberg, D. K., and M. R. Landry. 2017. Zooplankton and the ocean
carbon cycle. Annual Review of Marine Science \textbf{9}: 413--444.
doi:\href{https://doi.org/10.1146/annurev-marine-010814-015924}{10.1146/annurev-marine-010814-015924}

\leavevmode\vadjust pre{\hypertarget{ref-szeligowska2021}{}}%
Szeligowska, M., E. Trudnowska, R. Boehnke, A. M. Dąbrowska, K.
Dragańska-Deja, K. Deja, M. Darecki, and K. Błachowiak-Samołyk. 2021.
The interplay between plankton and particles in the Isfjorden waters
influenced by marine- and land-terminating glaciers. Science of The
Total Environment \textbf{780}: 146491.
doi:\href{https://doi.org/10.1016/j.scitotenv.2021.146491}{10.1016/j.scitotenv.2021.146491}

\leavevmode\vadjust pre{\hypertarget{ref-trudnowska2021}{}}%
Trudnowska, E., L. Lacour, M. Ardyna, A. Rogge, J. O. Irisson, A. M.
Waite, M. Babin, and L. Stemmann. 2021. Marine snow morphology
illuminates the evolution of phytoplankton blooms and determines their
subsequent vertical export. Nature Communications \textbf{12}: 2816.
doi:\href{https://doi.org/10.1038/s41467-021-22994-4}{10.1038/s41467-021-22994-4}

\leavevmode\vadjust pre{\hypertarget{ref-turner2004}{}}%
Turner, J. 2004. The importance of small planktonic copepods and their
roles in pelagic marine food webs,.

\leavevmode\vadjust pre{\hypertarget{ref-vilgrain2021}{}}%
Vilgrain, L., F. Maps, M. Picheral, M. Babin, C. Aubry, J.-O. Irisson,
and S.-D. Ayata. 2021. Trait-based approach using in situ copepod images
reveals contrasting ecological patterns across an Arctic ice melt zone.
Limnology and Oceanography \textbf{66}: 1155--1167.
doi:\href{https://doi.org/10.1002/lno.11672}{10.1002/lno.11672}

\leavevmode\vadjust pre{\hypertarget{ref-whitmore2021}{}}%
Whitmore, B. M., and M. D. Ohman. 2021. Zooglider-measured association
of zooplankton with the fine-scale vertical prey field. Limnology and
Oceanography \textbf{66}: 3811--3827.
doi:\href{https://doi.org/10.1002/lno.11920}{10.1002/lno.11920}

\leavevmode\vadjust pre{\hypertarget{ref-williamson2011}{}}%
Williamson, C. E., J. M. Fischer, S. M. Bollens, E. P. Overholt, and J.
K. Breckenridge. 2011. Toward a more comprehensive theory of zooplankton
diel vertical migration: Integrating ultraviolet radiation and water
transparency into the biotic paradigm. Limnology and Oceanography
\textbf{56}: 1603--1623.
doi:\href{https://doi.org/10.4319/lo.2011.56.5.1603}{10.4319/lo.2011.56.5.1603}

\leavevmode\vadjust pre{\hypertarget{ref-Zaret1976}{}}%
Zaret, T. M., and J. S. Suffern. 1976. Vertical migration in zooplankton
as a predator avoidance mechanism1. Limnology and Oceanography
\textbf{21}: 804--813.
doi:\href{https://doi.org/10.4319/lo.1976.21.6.0804}{10.4319/lo.1976.21.6.0804}

\end{CSLReferences}

\hypertarget{supplemental-information}{%
\section{Supplemental Information}\label{supplemental-information}}

\begin{figure}

{\centering \includegraphics{index_files/figure-pdf/unnamed-chunk-17-1.pdf}

}

\caption{Supplemental Figure S1. Map of CTD Cast Deployments. Dark
triangle points indicate night casts, tan circles indicate day casts.}

\end{figure}

\begin{figure*}

{\centering \includegraphics[width=1\textwidth,height=\textheight]{index_files/figure-pdf/unnamed-chunk-19-1.pdf}

}

\caption{Supplemental Figure S2. Physical parameters across individual
cruises. Vertical bars indicate CTD casts events with black indicating
night and grey indicating day.}

\end{figure*}

\begin{figure*}

{\centering \includegraphics[width=1\textwidth,height=\textheight]{index_files/figure-pdf/unnamed-chunk-19-2.pdf}

}

\caption{Supplemental Figure S2. Physical parameters across individual
cruises. Vertical bars indicate CTD casts events with black indicating
night and grey indicating day.}

\end{figure*}

\begin{figure*}

{\centering \includegraphics[width=1\textwidth,height=\textheight]{index_files/figure-pdf/unnamed-chunk-19-3.pdf}

}

\caption{Supplemental Figure S2. Physical parameters across individual
cruises. Vertical bars indicate CTD casts events with black indicating
night and grey indicating day.}

\end{figure*}

\begin{figure*}

{\centering \includegraphics[width=1\textwidth,height=\textheight]{index_files/figure-pdf/unnamed-chunk-19-4.pdf}

}

\caption{Supplemental Figure S2. Physical parameters across individual
cruises. Vertical bars indicate CTD casts events with black indicating
night and grey indicating day.}

\end{figure*}

\begin{figure*}

{\centering \includegraphics[width=1\textwidth,height=\textheight]{index_files/figure-pdf/unnamed-chunk-19-5.pdf}

}

\caption{Supplemental Figure S2. Physical parameters across individual
cruises. Vertical bars indicate CTD casts events with black indicating
night and grey indicating day.}

\end{figure*}

\begin{figure*}

{\centering \includegraphics[width=1\textwidth,height=\textheight]{index_files/figure-pdf/unnamed-chunk-19-6.pdf}

}

\caption{Supplemental Figure S2. Physical parameters across individual
cruises. Vertical bars indicate CTD casts events with black indicating
night and grey indicating day.}

\end{figure*}

\begin{figure*}

{\centering \includegraphics[width=1\textwidth,height=\textheight]{index_files/figure-pdf/unnamed-chunk-19-7.pdf}

}

\caption{Supplemental Figure S2. Physical parameters across individual
cruises. Vertical bars indicate CTD casts events with black indicating
night and grey indicating day.}

\end{figure*}

\begin{figure*}

{\centering \includegraphics[width=1\textwidth,height=\textheight]{index_files/figure-pdf/unnamed-chunk-19-8.pdf}

}

\caption{Supplemental Figure S2. Physical parameters across individual
cruises. Vertical bars indicate CTD casts events with black indicating
night and grey indicating day.}

\end{figure*}

\begin{figure*}

{\centering \includegraphics[width=1\textwidth,height=\textheight]{index_files/figure-pdf/unnamed-chunk-19-9.pdf}

}

\caption{Supplemental Figure S2. Physical parameters across individual
cruises. Vertical bars indicate CTD casts events with black indicating
night and grey indicating day.}

\end{figure*}

\begin{figure*}

{\centering \includegraphics[width=1\textwidth,height=\textheight]{index_files/figure-pdf/unnamed-chunk-19-10.pdf}

}

\caption{Supplemental Figure S2. Physical parameters across individual
cruises. Vertical bars indicate CTD casts events with black indicating
night and grey indicating day.}

\end{figure*}

\begin{figure*}

{\centering \includegraphics[width=1\textwidth,height=\textheight]{index_files/figure-pdf/unnamed-chunk-19-11.pdf}

}

\caption{Supplemental Figure S2. Physical parameters across individual
cruises. Vertical bars indicate CTD casts events with black indicating
night and grey indicating day.}

\end{figure*}

\begin{figure*}

{\centering \includegraphics[width=1\textwidth,height=\textheight]{index_files/figure-pdf/unnamed-chunk-19-12.pdf}

}

\caption{Supplemental Figure S2. Physical parameters across individual
cruises. Vertical bars indicate CTD casts events with black indicating
night and grey indicating day.}

\end{figure*}

\begin{figure*}

{\centering \includegraphics[width=1\textwidth,height=\textheight]{index_files/figure-pdf/unnamed-chunk-19-13.pdf}

}

\caption{Supplemental Figure S2. Physical parameters across individual
cruises. Vertical bars indicate CTD casts events with black indicating
night and grey indicating day.}

\end{figure*}

\begin{figure*}

{\centering \includegraphics[width=1\textwidth,height=\textheight]{index_files/figure-pdf/unnamed-chunk-19-14.pdf}

}

\caption{Supplemental Figure S2. Physical parameters across individual
cruises. Vertical bars indicate CTD casts events with black indicating
night and grey indicating day.}

\end{figure*}

\begin{figure*}

{\centering \includegraphics[width=1\textwidth,height=\textheight]{index_files/figure-pdf/unnamed-chunk-19-15.pdf}

}

\caption{Supplemental Figure S2. Physical parameters across individual
cruises. Vertical bars indicate CTD casts events with black indicating
night and grey indicating day.}

\end{figure*}

\begin{figure*}

{\centering \includegraphics[width=1\textwidth,height=\textheight]{index_files/figure-pdf/unnamed-chunk-19-16.pdf}

}

\caption{Supplemental Figure S2. Physical parameters across individual
cruises. Vertical bars indicate CTD casts events with black indicating
night and grey indicating day.}

\end{figure*}

\begin{figure*}

{\centering \includegraphics[width=1\textwidth,height=\textheight]{index_files/figure-pdf/unnamed-chunk-19-17.pdf}

}

\caption{Supplemental Figure S2. Physical parameters across individual
cruises. Vertical bars indicate CTD casts events with black indicating
night and grey indicating day.}

\end{figure*}

\begin{figure*}

{\centering \includegraphics[width=1\textwidth,height=\textheight]{index_files/figure-pdf/unnamed-chunk-19-18.pdf}

}

\caption{Supplemental Figure S2. Physical parameters across individual
cruises. Vertical bars indicate CTD casts events with black indicating
night and grey indicating day.}

\end{figure*}

\begin{figure*}

{\centering \includegraphics[width=1\textwidth,height=\textheight]{index_files/figure-pdf/unnamed-chunk-19-19.pdf}

}

\caption{Supplemental Figure S2. Physical parameters across individual
cruises. Vertical bars indicate CTD casts events with black indicating
night and grey indicating day.}

\end{figure*}

\begin{figure}

{\centering \includegraphics[width=1\textwidth,height=1\textheight]{index_files/figure-pdf/unnamed-chunk-21-1.pdf}

}

\caption{Supplemental Figure S3. PCA plot with major loading variables
plotted.}

\end{figure}

\begin{longtable}[]{@{}lrrrr@{}}
\toprule()
& Dim.1 & Dim.2 & Dim.3 & Dim.4 \\
\midrule()
\endhead
area & 0.7115131 & 0.5137738 & -0.1301928 & 0.3789753 \\
circ. & -0.7627265 & 0.2665557 & -0.3026239 & 0.3034329 \\
elongation & 0.4080632 & -0.1487560 & -0.7831847 & -0.2965671 \\
feret & 0.9096148 & 0.2756948 & -0.2479648 & 0.0082808 \\
fractal & 0.8816770 & 0.2904962 & -0.0440580 & 0.1793785 \\
histcum1 & 0.2197180 & -0.8987922 & -0.0901744 & 0.3005361 \\
major & 0.7593652 & 0.3432489 & -0.5115438 & 0.1498407 \\
mean & 0.2266145 & -0.9201880 & -0.0611832 & 0.2875560 \\
median & 0.1758116 & -0.8872673 & 0.0657406 & 0.1668198 \\
minor & 0.5668215 & 0.5593596 & 0.2891461 & 0.4585166 \\
perim. & 0.9268957 & 0.2685898 & 0.0681774 & 0.1135716 \\
perimferet & 0.3251295 & 0.0068372 & 0.7989292 & 0.2111105 \\
perimmajor & 0.4183142 & -0.1460583 & 0.8627707 & -0.1389696 \\
skew & -0.3216781 & 0.7017490 & 0.1514517 & -0.1447764 \\
stddev & -0.2234641 & 0.7971849 & 0.1531288 & -0.3647663 \\
symetriev & 0.7091737 & -0.2793593 & -0.0605051 & -0.5720602 \\
symetrievc & 0.5673730 & -0.1807542 & -0.2368157 & -0.4811256 \\
thickr & 0.4609473 & -0.2379533 & 0.5832035 & -0.3954358 \\
\bottomrule()
\end{longtable}

Supplemental Table S2. Loading scores for morphological factors on the
PCA.



\end{document}
